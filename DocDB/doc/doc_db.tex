\documentclass[12pt]{article}

\input{memo_setup}
\input{abbrev}
\usepackage{ewv_environ}      %   %% Define some new environments                            

\begin{document}                                                 

\title{BTeV Document Database Design and Interface}

\author{
        Eric W. Vaandering, Vanderbilt University\\
        Lynn Garren, Fermilab\\
        \\
        {BTeV-doc-140}
       }
        

\date{27 November 2001 \\
      Revised: 8 June 2002}

\maketitle

\begin{abstract} 
In this memo, we describe the design and implementation of the BTeV document
database.
\end{abstract}

\section{Design considerations}

BTeV needs a document database that satisfies a number of requirements:
\begin{itemize}
\item{A single database to store talks from group meetings,
conference talks and proceedings, and publications \emph{and} to present these
special cases in ways that makes sense for those special case.}
\item{All documents are kept on the BTeV web server, not personal machines. 
This way we don't suffer from link rot (linked-to files disappearing from the web).}
\item{Each document can have multiple revisions and old revisions are still
available.}
\item{Each revision of each document can have multiple files. This accommodates
multiple file types (source and  presentable) and/or child files (especially
useful for trees of HTML  documents).}
\item{The ability to upload documents by browser upload or forcing the 
document database to download via http.}
\end{itemize}

There are several security issues we want to resolve too:
\begin{itemize}
\item{The ability to not only have public and private documents but also 
documents that are accessible to subgroups (like RTES) which don't know 
the BTeV password. We also want to have documents that are only 
accessible to sub-groups (like the executive council).}
\item{Users should only know that a document exists if they have 
privileges to view it.}
\item{The ability to easily move documents, or just certain versions of 
documents, from restricted to public and vice-versa.}
\end{itemize}

The document database system consists of three pieces which work together: a
web-based user interface, a relational database server to store information
about the documents, and a file system to store the documents. 
The BTeV web server serves files from the file system to the users.

\section{Interface}

The document database is completely accessible via the web. HTML and HTML forms
provide the user interface and CGI scripts 
handle all user access to the underlying database, both for placing information in the
database and retrieving it.

All access and data entry scripts are written in Perl using the Perl::CGI  and 
Perl::DBI database access modules. Additional access programs
can be written using PHP, C++, or any other  language with an SQL interface. 


\subsection{Adding information to the database}

There are five different ways to add or update information in the database. Each is
used for a particular circumstance:
\begin{enumerate}
\item{Reserve a document \#: This is used for documents that don't yet exist, but a
number is needed. This allows the writer to know the document number before the real
document is placed in the database.}
\item{New document: This is for documents that have not been entered or reserved in
the database. This is the way the first version of a document is entered.}
\item{Update a document: This is the way updated versions of existing documents are
placed in the database. The document number remains the same, but the version number
is incremented.}
\item{Update database information: This is used when the document hasn't changed, but the
information about it has. For example, it's now published so that information has to
be added. Security settings on documents are also changed this way. The version
number is unchanged.}
\item{Add files to a document: This option is used to add new files to an existing
version of a document, perhaps files that were forgotten or an additional presentation
format of an old document.  If the content of the document has changed at all, the
Update, not the Add file, option should be used. The version
number is unchanged.}
\end{enumerate}

Retrieved information from the database is presented in two ways, lists of
documents that meet specific criteria and a display of all the information
available about a document. The information presented in the list is intended to
provide enough information for the user to decide if they want more information.

Currently, documents can be listed by author, subtopic, type, and date of last
modification.  While a more comprehensive search engine is not yet included in the
product, one is planned.

\subsection{Security}

When you access any part of the interface, you will be asked by your web browser
to supply a username and password. Which username you use will influence which
documents you have access too. You will not be able to see any information about
any documents you can't access.

\section{Database}

The database design is essentially finished, although small additions will
probably be needed for refereed publications (see
\secref{extend}).

The document 
database is implemented as a number of different tables. Because each 
document can have multiple revisions and each revision can have 
multiple files (and authors and topics), there are three main tables 
Document, DocumentRevision, and DocumentFile. The hierarchy of these 
tables, their elements, and the auxiliary tables are shown in
\figref{db_structure}.

\begin{figure}
\begin{center}
\includegraphics[width=6in]{DocDB_model.eps}
\caption{The design of the document database. Indexed fields are shown in red and the
symbols are described in the text.}
\figlabel{db_structure}
\end{center}
\end{figure}

\subsection{Database design considerations}

The database design we are using is in the ``third normal form.'' For a longer
discussion of what this means and how a database is properly designed
see~\cite{YargerReeseKing:1999}. Essentially it boils down to three requirements:
\begin{enumerate}
\item{Each field of a database contains only one piece of information}
\item{Any field which has values in common across multiple entries should use a
unique identifier and another table}
\item{No two fields in a table should depend on each other}
\end{enumerate}
To take a simple example, the field containing the document type should contain
a unique identifier, not the name of the type because the name might have been
misspelled (rule 2). The institution of the person who wrote the document
something is not a property of the document, but of the person (rule 3).

The ``tree''-like symbols indicate the one-to-many relationships within the
database. For instance, there can (and will) be many revisions of a  single
document.  Rule 1 mandates that there can be no ``trees'' with branches on both
ends. For example, because each document revision can have multiple authors and
an author can be associated with multiple documents, an intermediate table
RevisionAuthor is used. There is one entry in this table for each author of
each document revision.

Fields shown in red are indexed for fast look up. Other fields must be searched
on which is more time consuming. Each index takes up some  amount of disk
space, so the number of indices should be kept to a reasonable number. The ID
numbers in each entry indicate what other entries they are associated with. In
addition, each table has a timestamp which is not shown.

\subsection{Explanation of tables and fields}

While most of the tables and fields in \figref{db_structure} should be self
explanatory, a few should be described further.

\paragraph{Document:} It might be surprising that very little information is
stored about the actual document (just who requested it and the document type).
This is because most of the information about a document can change and is
properly stored with each document revision rather than the document.

\paragraph{DocumentRevision:} This is where most of the information about a
document is stored. The Obsolete field is set when the database information is
updated but the document itself is not. A new version number is not generated in
this case, so the DocumentID and VersionNumber fields are not sufficient to
specify a unique DocRevID.

\paragraph{DocumentFile:} This table stores  the information about each file in a
document revision. RootFile is used to distinguish between primary and supporting
files within the document. Under some circumstances only the main files are listed
or the two types are segregated.

\paragraph{Linking tables:} The tables RevisionSecurity, RevisionTopic, and
RevisionAuthor are all used to link multiple settings for security, topics, and
authors to a single document revision.

\paragraph{Topics:} Topics are divided into MajorTopics and MinorTopics (or
sub-topics). Each minor topic is associated with a single major topic. The
document revision is associated with one or more minor topics, not major topics. 

\paragraph{Conference:} For conferences, the short and long names of the
conference are stored in MinorTopic. Additional information about the
conference, such as location and dates are stored in the Conference table. Each
row from conference is linked to a row from MinorTopic.

\paragraph{SecurityGroups and GroupHierarchy:} Each document revision can be
viewable by an arbitrary number of groups. A group can be the collaboration as
whole, a sub-group, or a group associated with BTeV like the Real Time Embedded
Systems (RTES) group. The table GroupHierarchy implements a hierarchy of groups
within the system. For instance, BTeV is currently defined as a ``parent'' of
``RTES'' so that those with BTeV access can read all documents under the RTES
security, but not vice versa. This is primarily a convenience measure.


\subsection{Implementation}

This database is implemented using a MySQL server running on
\url{fnsimu1.fnal.gov}.  The use of non-standard SQL features in the interface
has been kept to a minimum to ease porting to another another database if needed.
(MySQL has a very useful but non-standard function \texttt{last\_insertid} which
returns the unique identifier of the last item inserted into the database. We use
this because it is safer and more elegant than keeping track of all these numbers
in a separate table.)

\section{File system and security}

All documents are placed into a single directory tree. There is no  division
based on security or topic since these are fluid designations  (and each
document can have multiple values for these settings). The  file system is
arranged only by document number. Each revision of each  document is placed in
its own directory, which means that most  directories will only have a small
number of files. While this is somewhat inefficient, the  advantage of this
scheme is two-fold. First, each of the files of a  particular revision need not
be renamed. Second, web access to each  revision is controlled with a
potentially unique .htaccess file. (These .htaccess files are written by the
interface software at the time files placed in the file system.)

Once the database system has placed a document in the file system, it has no
further interaction with the file. (It never checks to make sure it is still
there.) From this point on, sending the actual files to the user is the sole
responsibility of the web server.

To avoid possible issues with large numbers of files in a single 
directory, the document directories are further divided into groups of 
100. For example, imagine that the 3rd revision of the 1029th document 
contains one file, text.ps. Then the full path of the file is:

\texttt{\$DOCROOT/0010/001029/003/text.ps}

As you can see, the file system is designed for up to 999,999 documents 
each of which can have up to 999 versions.

\section{Special Types of Documents}\seclabel{extend}

In addition to generic type of documents, extra information exists about
documents which are associated with group meetings, conferences, or refereed
journals. With small extensions to the database structure, this additional
information can be easily captured and represented.

\subsection{Group meeting talks}

Up until this point, talks given at group meetings outnumber the total number
of documents entered into the old document database. We would like to
completely integrate the group meeting talks into the document database. They
should receive special treatment since they will be a large fraction of the
documents and people have certain expectations of how the information is
presented.

Associating a talk with a meeting is done by choosing, as one of the document's
subtopics, the correct meeting. Then a search on that topic will return all the
talks for the meeting. Additionally, meetings are sorted in reverse
chronological order rather than alphabetically wherever they appear.

Making a usable list of meeting talks involves making a minor change to the
listing format by listing the files associated with a talk rather than the
modification date. 

Together these changes yield an interface that is similar to what collaborators
are used to. Combined with the instant update nature of the database and its
indexing capabilities should make the overall experience much better.

\subsection{Conference talks and proceedings}

Conference talks and proceedings pose a slightly different problem. As with
group meetings, associating a document with a conference will be done by
choosing an appropriate sub-topic. Listing of documents will be by conference
date, not modification date, in reverse chronological order. 

A full listing, however, should also contain information about the conference
itself such as location, dates, a URL, etc. This was accomplished by adding
an additional table to the database linked to the conference's MinorTopicID.

This functionality has been partially implemented. Listing by conference is
possible, with slightly different options. The sort is currently done on the
modification rather than conference date. 

Yet to be decided is the issue of how to handle a conference talk and its
associated proceedings. Are these filed as two documents or just one? If just
one, what document type is to be used?

\subsection{Journal publications}

Publications in journals have a slightly enhanced interface as well.
While we have a Publication Information field in the database, what is
placed there is completely up to the user and is intended more as a place to put
notes rather than information in a well defined form. 

The publication information field will remain to allow arbitrary notes to be
added by the users.

To support journal publications,additional fields have been added to
the database to contain this information in a very well defined manner
for a set of predefined journals. This will allow a coherent presentation of a
list of publications and advanced features such as links to the publisher's web
site, automatic generation of BibTeX entries, etc.

This is also partially implemented. The data is collected and stored in the
database. It is also displayed on the full document view page, but there is no
code written to just list public documents and their references, etc.

\bibliographystyle{phaip}
\bibliography{physjabb,abbrev,other,btev}                                                                                      

\end{document}




